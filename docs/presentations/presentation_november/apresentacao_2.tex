\documentclass[xcolor=dvipsnames,10pt,aspectratio=169]{beamer}
%\documentclass[xcolor=dvipsnames,10pt]{beamer}
\usepackage{etex}
\usepackage{pgf,pgfarrows,pgfnodes,pgfautomata,pgfheaps,pgfshade}
\usepackage[absolute,overlay]{textpos} 
%\usepackage{algorithm}
\usepackage{amsmath,amssymb}
\usepackage[utf8]{inputenc} 
\usepackage{colortbl}
\usepackage{graphicx} 
\usepackage[english]{babel}
\usepackage{tabularx} 
\usepackage{multirow}
\usepackage{booktabs}
\usepackage{listings}
%\usepackage{multimedia}
\usepackage{animate}
\usepackage{xcolor}
\usepackage{array}
\usepackage{longtable}
\usepackage{makecell}
\usepackage{caption}
\usetheme{Madrid} 
\usepackage{amsmath}
\usepackage{movie15}
\usepackage{tikz}
\usetikzlibrary{shapes.geometric,arrows,shadows}

\lstset{ %
%	backgroundcolor=\color{white},   % choose the background color; you must add \usepackage{color} or \usepackage{xcolor}
%	basicstyle=\footnotesize,        % the size of the fonts that are used for the code
	basicstyle=\scriptsize,        % the size of the fonts that are used for the code
	breakatwhitespace=false,         % sets if automatic breaks should only happen at whitespace
	breaklines=true,                 % sets automatic line breaking
	captionpos=t,                    % sets the caption-position to bottom
	commentstyle=\color{mygreen},    % comment style
	deletekeywords={...},            % if you want to delete keywords from the given language
	escapeinside={\%*}{*)},          % if you want to add LaTeX within your code
	extendedchars=true,              % lets you use non-ASCII characters; for 8-bits encodings only, does not work with UTF-8
%	frame=single,                    % adds a frame around the code
	keepspaces=true,                 % keeps spaces in text, useful for keeping indentation of code (possibly needs columns=flexible)
	keywordstyle=\color{blue},       % keyword style
%	language=make,                 % the language of the code
	morekeywords={*,...},            % if you want to add more keywords to the set
%	numbers=left,                    % where to put the line-numbers; possible values are (none, left, right)
%	numbersep=5pt,                   % how far the line-numbers are from the code
	numberstyle=\tiny\color{mygray}, % the style that is used for the line-numbers
	rulecolor=\color{black},         % if not set, the frame-color may be changed on line-breaks within not-black text (e.g. comments (green here))
	showspaces=false,                % show spaces everywhere adding particular underscores; it overrides 'showstringspaces'
	showstringspaces=false,          % underline spaces within strings only
	showtabs=false,                  % show tabs within strings adding particular underscores
	stepnumber=2,                    % the step between two line-numbers. If it's 1, each line will be numbered
}

\definecolor{mygreen}{rgb}{0,0.6,0}
\definecolor{mygray}{rgb}{0.5,0.5,0.5}
\definecolor{mymauve}{rgb}{0.58,0,0.82}

\usecolortheme{beaver}
\newcommand{\ul}{\underline}
\setbeamertemplate{footline}{\scriptsize{\vspace*{0.3cm}\hspace*{15cm}\insertframenumber\,/\,\inserttotalframenumber}}
\setbeamertemplate{caption}[numbered]
\setbeamerfont{caption}{size=\fontsize{8}{5}}

\setbeamercolor{block title}{	bg=Sepia , fg = White}
\setbeamercolor{block body}{bg=Brown!15, fg=Sepia }
\setbeamercolor{item projected}{bg=Sepia, fg=White}
\setbeamercolor{number projected}{bg = Black}

%declara as imagens usadas no layout do slide
\pgfdeclareimage[height=0.8cm]{mflab}{figuras/logo_mflab_transparente.png}
\pgfdeclareimage[height=1.0cm]{logoufu}{figuras/logo_ufu.jpg}
\pgfdeclareimage[height=1.0cm]{petro}{figuras/petrobras_2.png}

%posiciona o logotipo do MFLab
\setlength{\TPHorizModule}{1mm}
\setlength{\TPVertModule}{1mm}
\newcommand{\placelogomflab} 
{ 
	\begin{textblock}{13}(150.0,0.0)
		\pgfuseimage{mflab} 
	\end{textblock} 
	
	\begin{textblock}{13}(150.0,70.0)
		\pgfuseimage{petro} 
	\end{textblock} 
}
\setlength{\TPHorizModule}{1mm}
\setlength{\TPVertModule}{1mm}
\newcommand{\placelogo} 
{ 
	\begin{textblock}{13}(150.0,0.0)
		\pgfuseimage{mflab} 
	\end{textblock} 
	
	\begin{textblock}{13}(0.0,80.0)
		\pgfuseimage{petro} 
	\end{textblock} 
}

\title{NEURAL NETWORK ASSISTED CAVITY CONVECTION FLOW}

\author{ Felipe J. O. Ribeiro \\ \and \\ Prof. Dr. Aristeu da Silveira Neto \\ Prof. Dr. Aldemir Aparecido Cavallini Junior}

\date{\tiny{\today}}
\newcolumntype{C}[1]{>{\centering\let\newline\\\arraybackslash\hspace{0pt}}m{#1}}


\begin{document}

\begin{frame}\placelogomflab
	\frametitle 
	{ \vfill
		\centering
		{
		\small{Federal University of Uberlândia}\\
			% \small{Programa de Pós-Graduação em Engenharia Mecânica}\\
		\small{Fluid Mechanics laboratory}\\
		}
	}
	\maketitle
\end{frame}

\section<presentation>*{Index}
	
\begin{frame}
	\frametitle{Index}\placelogomflab 
	{\scriptsize \tableofcontents}
\end{frame}

\AtBeginSection[]
{
 \begin{frame}<beamer>
  \frametitle{Index}\placelogomflab 
  {\scriptsize \tableofcontents[current,currentsection]}
 \end{frame}
}

\AtBeginSubsection[]
{
 \begin{frame}<beamer>
  \frametitle{Index}\placelogomflab 
  {\scriptsize \tableofcontents[current,currentsubsection]}
 \end{frame}
}


\section{Development}

\begin{frame}\frametitle{Objectives}
	\centering
	In this segment we will discuss about the preassure field data from the simulations, seeking to stablish how to study forecasting targets for our optimized algorithm.
	\vspace{0.5cm}

	\flushleft
	The core topics of this study are:\\
	\quad $\bullet$ Getting preassure data from simulations;\\
	\quad $\bullet$ Using Kriging for pressure forecasting;\\
	\quad $\bullet$ Testing various sampling methods for the learning process;\\
\end{frame}


\begin{frame}\frametitle{About the preassure acquisition}
	\begin{minipage}[h!]{0.49\textwidth}
		The first implementation that had to be done was the data import module. The routine was configured to write in a way that paraview could understand as as such was created with a very strict formatation.
		With python, a module was made to capture the preassure of a target cell.
	\end{minipage}
	\begin{minipage}[h!]{0.5\textwidth}
		\begin{figure}
			\centering
	 		\animategraphics[scale = 0.25 , loop , autoplay]{10}{figuras/p_result/frame-}{0}{250}
	 		\caption{Pressure domain solution.}
		\end{figure}
	\end{minipage}
\end{frame}


\begin{frame}\frametitle{Examples of probes}
	\begin{minipage}[h!]{0.49\textwidth}
		\begin{figure}
			\centering
			\includegraphics[clip=true, scale=0.25]{figuras/preassure_15_15.png}
	 		\caption{Pressure probe ate the position (15,15).}
		\end{figure}
	\end{minipage}
	\begin{minipage}[h!]{0.5\textwidth}
		\begin{figure}
			\centering
			\includegraphics[clip=true, scale=0.25]{figuras/preassure_1_29.png}
	 		\caption{Pressure probe ate the position (1,29).}
		\end{figure}
	\end{minipage}
\end{frame}


\begin{frame}\frametitle{Training based on same cell through time}
	\begin{minipage}[h!]{0.49\textwidth}
		$\bullet$ First, with the simulation on the side, data was taken and used for training Kriging (KPLS) methods of forecasting;\\
		$\bullet$ The first sampling method was based on the same cell, through time;\\
	\end{minipage}
	\begin{minipage}[h!]{0.5\textwidth}
		\begin{figure}

			\centering
			\includegraphics[clip=true, scale=2]{figuras/time_step_pressure_1.png}
	 		\caption{Pressure on step i is based on pressure in step i-1. Same cell. Velocity independent.}
		\end{figure}
	\end{minipage}
\end{frame}

\begin{frame}\frametitle{Results with recursive Kriging through time.}
	\begin{minipage}[h!]{0.49\textwidth}
		\begin{figure}
			\centering
			\includegraphics[clip=true, scale=0.25]{figuras/preassure_15_15_krigin_100.png}
	 		\caption{Pressure on step i is based on pressure in step i-1. Same cell. Velocity independent. On(15,15).}
		\end{figure}
	\end{minipage}
	\begin{minipage}[h!]{0.5\textwidth}
		\begin{figure}
			\centering
			\includegraphics[clip=true, scale=0.25]{figuras/preassure_1_29_krigin_100.png}
	 		\caption{Pressure on step i is based on pressure in step i-1. Same cell. Velocity independent. On (1,29).}
		\end{figure}
	\end{minipage}
\end{frame}


\begin{frame}\frametitle{Testing with non-recursive Kriging.}
	The results were not accurate. Thinking that it could be an implementation error i tryied not use the recursive solution.\\
	\begin{minipage}[h!]{0.49\textwidth}
		\begin{figure}
			\centering
			\includegraphics[clip=true, scale=0.25]{figuras/preassure_15_15_krigin_0.png}
	 		\caption{Pressure on step i is based on pressure in step i-1. Same cell. Velocity independent. On(15,15).}
		\end{figure}
	\end{minipage}
	\begin{minipage}[h!]{0.5\textwidth}
		\begin{figure}
			\centering
			\includegraphics[clip=true, scale=0.25]{figuras/preassure_1_29_krigin_0.png}
	 		\caption{Pressure on step i is based on pressure in step i-1. Same cell. Velocity independent. On (1,29).}
		\end{figure}
	\end{minipage}
\end{frame}

\begin{frame}\frametitle{Testing with partial non-recursive Kriging.}
	\begin{minipage}[h!]{0.49\textwidth}
		\begin{figure}
			\centering
			\includegraphics[clip=true, scale=0.25]{figuras/preassure_15_15_krigin_50.png}
	 		\caption{Pressure on step i is based on pressure in step i-1. Same cell. Velocity independent. On(15,15).}
		\end{figure}
	\end{minipage}
	\begin{minipage}[h!]{0.5\textwidth}
		\begin{figure}
			\centering
			\includegraphics[clip=true, scale=0.25]{figuras/preassure_1_29_krigin_50.png}
	 		\caption{Pressure on step i is based on pressure in step i-1. Same cell. Velocity independent. On (1,29).}
		\end{figure}
	\end{minipage}
\end{frame}


\section{Acknowledgments}
	\begin{frame}
		\placelogomflab 
		\frametitle{Acknowledgments}
		\begin{figure}
		\begin{center}
		\begin{tabular}{c c}
			{\includegraphics[trim=0.0cm 0.0cm 0.0cm 0.0cm,clip=true,height=0.2\textheight]{figuras/petrobras.png}}&{\includegraphics[trim=0.0cm 0.0cm 0.0cm 0.0cm,clip=true,height=0.2\textheight]{figuras/logo_mflab.png}}\\
			{\includegraphics[trim=0.0cm 0.0cm 0.0cm 0.0cm,clip=true,height=0.2\textheight]{figuras/cnpq.png}}&{\includegraphics[trim=0.0cm 0.0cm 0.0cm 0.0cm,clip=true,height=0.2\textheight]{figuras/CAPES.png}}\\
			{\includegraphics[trim=0.0cm 0.0cm 0.0cm 0.0cm,clip=true,height=0.2\textheight]{figuras/FAPEMIG.jpg}}&{\includegraphics[trim=0.0cm 0.0cm 0.0cm 0.0cm,clip=true,height=0.2\textheight]{figuras/UFU_black.jpg}}\\
		\end{tabular}
		\end{center}
		\end{figure}
	\end{frame}

	\begin{frame}
		\placelogomflab 
		\frametitle{Acknowledgments}
		\fontsize{44pt}{7.2}\selectfont
		\begin{center}
			Thank you.
		\end{center}
	\end{frame}
\end{document}
